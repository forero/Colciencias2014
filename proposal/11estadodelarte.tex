
\section{Estado del arte}
% Revisión actual de la temática en el contexto nacional e
% internacional, avances, desarrollos y tendencias. 

A contnuaci\'on revisamos el estado del arte de las tres \'areas tem\'aticas
planteadas en el Marco Conceptual

Con la puesta en marcha de ambiciosos programas observacionales para
cuantificar la Ener\'gia Oscura durante la siguiente d\'ecada, los
esfuerzos te\'oricos se han  multiplicado para encontrar formas de
medir la expansi\'on acelerada con alta precisi\'on. Existen varios
m\'etodos observacionales utilizados en la actualidad para hacer este
ripo de mediciones. Entre ellos se destacan:

\begin{itemize}
\item La escala de la Oscilaci\'on Ac\'ustica de Bariones.
\item El crecimiento de estructura a trav\'es de las Distorsiones en
  el Espacio de Redshift.
\item El bosque Lyman-$\alpha$ de galaxias distantes.
\item Mediciones de distancias con supernovas.
\item Abundancia de c\'umulos de galaxias.
\end{itemize}


En los m\'etodos
que utilizan la estructura a gran escala se destacan, como
mencion\'abamos m\'as arriba, la medici\'on de la Oscilaci\'on
Ac\'ustica de Bariones (OAB) y la estimaci\'on de las Distorsiones en
el Espacio de Redshift (DER).

Dentro de las misiones observacionales.





\subsection{Simulaciones}


\subsection{Observaciones}

Las observaciones de 

