
\section{Marco conceptual}
% Aspectos conceptuales y teóricos que contextualicen el problema de
% investigación en una temática; así como otros aspectos que sean
% pertinentes a juicio de los proponentes. 


El principal elemento conceptual de este proyecto es la {\bf Cosmolog\'ia
vista a partir de la estructura del Universo a gran escala}. Este
elemento central se divide en tres temas: teor\'ia,
simulaciones y observaciones.



\subsection{Teor\'ia}

Desde el punto de vista te\'orico hay dos observables principales que
dependen de los par\'ametros cosmol\'ogicos: la historia de
expansi\'on del Universo y el crecimiento de la estructura a gran
escala. Ambos pueden ser medidos a partir de la distribuci\'on de
galaxias en el Universo.

La {\bf historia de expansi\'on} del Universo, medida por la constante
de Hubble dependiente del redshift H(z), dependende directamente del
contenido de mater\'ia y energ\'ia del Universo. Este contenido se 
puede separar en las sigientes componentes: la densidad de materia
bari\'onica ($\Omega_b$), la densidad de materia oscura
($\Omega_{dm}$) y la densidd de energ\'ia  oscura  ($\Omega_{DE}$), la
cual puede variar en el tiempo dependiendo del valor de la constante
$w$ en la ecuaci\'on de estado, $\Omega_{DE}$ donde 

\begin{equation}
\Omega_{DE}(z) = \Omega_{DE,0}(1+z)^{3(1+w)}.
\end{equation}

Para esto hemos usado el resultado de un Universo plano con
$\Omega_b+\Omega_{dm}+\Omega_{DE}=1$ y $w$ constante en el tiempo.
Tambi\'en hemos tomado $\Omega_{DE}$ de manera general para una
Energ\'ia Oscura sin asumir que se trata de una constante
cosmol\'ogica, $\Omega_\Lambda$ con $w=-1$. 

Hoy en d\'ia estos par\'ametros cosmol\'ogicos es\'an
acotados observacionalmente cerca a los siguientes valores $\Omega_b=0.05$,
$\Omega_{dm}=0.25$, $\Omega_{DE}=0.70$ y $w=-1$. De estos
par\'ametros los que m\'as inter\'es generan actualmente son
$\Omega_{DE}$ y $w$ porque est\'an relacionados con  la expansi\'on
acelerada del Universo, algo que puede corresponder a una simple constante
cosmol\'ogica o puede ser la evidencia de que Relatividad General no
es la teor\'ia correcta para la gravedad 
\cite{2014arXiv1401.0046M}.

Por otro lado, el {\bf crecimiento de estructura a gran escala} depende
principalmente de la fluctuaciones iniciales en el campo de densidad
de materia en las \'epocas tempranas del Universo y de la teor\'ia de
la gravedad que hace que esas fluctuaciones se amplifiquen generando
la variedad de estructura que observamos en el Universo. En este
contexto la cantidad central es el contraste de densidad $\delta({\bf
  x},t)\equiv\rho({\bf r},t)/\bar{\rho(r)}-1$ de la materia oscura. En
el r\'egimen lineal de crecimiento est\'a dado por la siguiente
ecuaci\'on diferencial 

\begin{equation}
\ddot{D} + 2H(z)\dot{D}- \frac{3}{2}\Omega_mH_{0}^2(1+z)^3D=0,
\end{equation}
%
donde $H_0$ es la constante de Hubble en el presente y $D(t)$ es la
funci\'on de crecimiento. Incluso en el r\'egimen no lineal el
contraste de densidad tambi\'en depende de esta funci\'on $D(t)$.  

Lo interesante en este caso es que diferentes modelos de la gravedad
(entre ellos la Relatividad General) hacen predicciones sobre el
comportamiento de esta funci\'on $D(t)$. De esta manera una estrategia
posible para probar teorias de gravedad modificada es medir al mismo
tiempo la historia de expansi\'on y la historia de crecimiento de
estructuras para ver si dan valores consistentes de $H(z)$ y de
$w$ \cite{2014arXiv1401.0046M}. 

\subsection{Simulaciones}

Aunque la decripci\'on anal\'itica esbozada anteriorment
permite acercarse a varios fen\'omenos del crecimiento de estructura,
una descripci\'on detallada solamente ha sido posible a trav\'es de
las simulaciones computacionales. Las simulaciones buscan seguir la
evoluci\'on del campo de densidad de materia oscura en funci\'on del
tiempo. 

En el caso de simulaciones en la cosmolog\'ia est\'andar $\Lambda$
Cold Dark Matter ($\Lambda$CDM) se sigue un aproximaci\'on en la cual
solamente se simula la componente oscura y no la bari\'onica, dado que
la primera domina la din\'amica del sistema. Luego de esto se toma un
volumen computacional que represente una regi\'on del universo con
distribuci\'on de materia oscura se discrerizada en part\'iculas
computacionales con posiciones y velocidades que pueden evolucionar en
el tiempo.  Esta forma general de realizar los c\'alculos ha sido
perfeccionada durante los \'ultimos treinta a\~nos de investigaci\'on
en el tema de formaci\'on de estructura en un Universo domiando por la
materia oscura
\cite{1985ApJ...292..371D,1999ApJ...522...82K,2005Natur.435..629S}.
Aunque es posible incluir una descripci\'on del gas bari\'onico y de
los procesos asociados a las formaci\'on estelar, nosotros centraremos
nuestra discusi\'on en torno a las simulaciones que solamente incluyen
la materia oscura, dado que son suficientes para los objetivos de
nuestro proyecto.

Como en todo experimento num\'erico
hay dos cantidades centrales para la descripci\'on del sistema: las
condiciones iniciales y las reglas para la evoluci\'on temporal del
sistema.  Las condiciones iniciales est\'an definen las
posiciones y velocidades iniciales de las part\'iculas
computacionales. En este caso los desplazamientos de las part\'iculas
con respecto a una distribuci\'on homog\'ena de part\'iculas est\'an
relacionados con las fluctuaciones tempranas en el campo de densidad y
se pueden describir estad\'isiticamente a trav\'es del espectro de
potencias. Una vez las posiciones iniciales est\'an determinadas se
determinan las velocidades iniciales, usualmente a trav\'es de la
aproximaci\'on de Zeldovich aunque es posible imponer estas
condiciones iniciales a trav\'es de otras aproximaciones perturbativas
\cite{2014MNRAS.439.3630W}. 

La simulaci\'on sigue entonces la evoluci\'on de las posiciones y
velocidades de las part\'iculas durante la historia del
Universo. En el rango de evoluci\'on no lineal se forman
sobre-densidades de materia oscura que reciben el nombre de halos, en
los centros de estos halos las galaxias deberian formarse y
evolucionar. En este punto es posible entonces pasar a una
descripci\'on de la distribuci\'on de la simulaci\'on en t\'erminos de
las posiciones, velocidades y masas de estos {\bf halos de materia
  oscura}, que son la cantidades de inter\'es al momento de vincular
las predicciones de los modelos con las observaciones. Estos halos
sirven en las construcci\'on de {\bf cat\'alogos ficticios} que se
son comparables con mapas de la distribuci\'on de galaxias en el
Universo.

Para la a relizaci\'on de simulaciones  que incluyan
efectos de diferentes modelos de energ\'ia oscura hay tres
posibilidades: campos homog\'eneos de energ\'ia oscura, campos
inhomog\'eneos de energ\'ia oscura y modelos con de inhomogeneidad a
gran escala \cite{2012PDU.....1..162B}. En este proyecto nos
centraremos en los modelos $\Lambda$CDM y en los homog\'eneos de energ\'ia
oscura que se pueden incluir en simulaciones hechas con la maquinaria
$\Lambda$CDM a trav\'es de una parametrizaci\'on para la evoluci\'on
del t\'ermino $\Omega_{DE}(z)$.

\subsection{Observaciones}

Las observaciones del Universo a gran escala empezaron a tener un rol
central en la astronom\'ia observacional y en la cosmolog\'ia con dos
colaboraciones: el Sloan Digital Sky Survey (SDSS) \cite{SDSS} y el Two Degree
Field Galaxy Redshift Survey (2dFGRS) \cite{2dF}. El objetivo
principal de estas campa\~nas observacionales fue el de tomar
espectros de cerca de un mill\'on de galaxias del Universo local sobre
una gr\'an \'area del cielo. A partir de estas observaciones de la
estructura del Universo a gran escala se pueden inferir diferentes
par\'ametros cosmol\'ogicos.

Dentro de las mediciones m\'as importantes a partir de estas dos misiones
ha sido el de la densidad de materia cosmol\'ogica $\Omega_m$
\cite{2001Natur.410..169P} y el pico en la funci\'on de
correlaci\'on correspondiente a la Oscilacion Ac\'ustica de Bariones
(OAB) \cite{Eisenstein2005}. Actualmente las observaciones recientes de la OAB
son las que proveeen una de las form\'as mas competitivas de acotar la
historia de expansi\'on del Universo \cite{2014MNRAS.441...24A}.

En cuanto a las mediciones del crecimiento de estructura (cotas sobre
la forma de la funci\'on $D(t)$ mencionada en la secci\'on sobre
teor\'ia) estas se han a trav\'es de la cuantificaci\'on de la
distrbuci\'on de las posiciones de las galaxias en la direcci\'on
radial y perpendicular al observador \cite{2014MNRAS.439.3504S},
aunque los resultados son de una precisi\'on que todav\'ia no permite
distinguir claramente entre el modelo est\'andar y varias teor\'ias
modificadas de la gravedad.
