\documentclass[spanish,notitlepage,letterpaper,11pt]{article} % para
                                % articulo en castellano
\usepackage[utf8]{inputenc}
%\usepackage[ansinew]{inputenc} % Acepta caracteres en castellano
\usepackage[spanish]{babel}    % silabea palabras castellanas
\usepackage{amsmath}
\usepackage{amsfonts}
\usepackage{amssymb}
\usepackage[colorlinks=true,urlcolor=blue,linkcolor=blue]{hyperref} % navega por el doc
\usepackage{graphicx}
\usepackage{geometry}           % See geometry.pdf to learn the layout options.
\geometry{letterpaper}          % ... or a4paper or a5paper or ... 
%\geometry{landscape}           % Activate for for rotated page geometry
%\usepackage[parfill]{parskip}  % Activate to begin paragraphs with an empty line rather than an indent
\usepackage{epstopdf}
\usepackage{fancyhdr} % encabezados y pies de pg
\pagestyle{fancy} 
\chead{} 
%\lhead{\textit{ Encabezado izquierdo }} % si se omite coloca el nombre de la seccion
%\rhead{\textbf{Encabezado derecho}} 
%\rfoot{\thepage} 

\voffset = -0.25in 
\textheight = 8.0in 
\textwidth = 6.5in
\oddsidemargin = 0.in
\headheight = 20pt 
\headwidth = 6.5in
\renewcommand{\headrulewidth}{0.5pt}
\renewcommand{\footrulewidth}{0,5pt}

\newcommand{\apj}{ApJ}
\newcommand{\apjs}{ApJS}
\newcommand{\apjl}{ApJL}
\newcommand{\aj}{AJ}
\newcommand{\mnras}{MNRAS}
\newcommand{\mnrassub}{MNRAS accepted}
\newcommand{\aap}{A\&A}
\newcommand{\aaps}{A\&AS}
\newcommand{\araa}{ARA\&A}
\newcommand{\nat}{Nature}
\newcommand{\prd}{PRD}
\newcommand{\physrep}{PhR}
\newcommand{\pasp}{PASP}
\newcommand{\pasj}{PASJ} 
\title{Proyecto:\\{\bf Cosmolog\'ia Computacional y Observacional}}
\author{Investigador Principal:\\{\bf Jaime Ernesto Forero Romero,
    PhD}\\
}
\begin{document}


\input{1titulo.tex}
\input{2investigadores.tex}



\newpage 
\tableofcontents 
\section{Conformaci\'on del equipo de investigaci\'onn}
%Colocar el nombre y código, registrado en el GrupLac, del o de los
%grupos de investigación. Al igual que el nombre de los demás
%integrantes que conforman el equipo de trabajo. Se debe incluir el
%tiempo de dedicación y funciones en el marco del proyecto.  

La investigaci\'on se har\'a dentro del grupo de astrof\'isica de la
Universidad de los Andes, c\'odigo GrupLAC COL0015473. Los integrantes
del grupo son los siguientes. 


\begin{itemize}
\item Un profesor de planta del grupo de Astrof\'isica de la
  Universidad de los Andes: Jaime Ernesto  Forero Romero, PhD.
  Dedicaci\'on: 7 horas/semana.
\item Dos estudiantes de doctorado en el departamento de F\'isica de
  la Universidad de los Andes: Felipe Leonardo G\'omez Cort\'es
  (F\'isico) y otro por   definir.  Dedicaci\'on: 20 horas / semana. 
\end{itemize}

\noindent
Con el apoyo de los siguientes asesores internacionales:

\begin{itemize}

\item Stefan Gottloeber, PhD. Cient\'ifico en el Leibniz Institute for
  Astrophysics, Alemania.  
\item Changbom Park, PhD. Miembro permanente del Korean Institute for
  Advanced Studies, Corea del Sur. 
\item Robert Cahn, PhD. Cient\'ifico en el Lawrence Berkeley National
  Laboratory, Estados   Unidos. 
\end{itemize}

\noindent
Adicionalmente el proyecto cuenta con el soporte del siguiente
personal de apoyo de la Universidad de los Andes 

\begin{itemize}
\item{Ingeniero de c\'omputo del departamento de F\'isica.}
\item{Ingeniero de c\'omputo de alto rendimiento en el Direcci\'on de
  Servicios de Informaci\'on y Tecnolog\'ia.}
\end{itemize}

\input{4antecedentes.tex}
\input{5tematica.tex}
\input{6resumenejecutivo.tex}



\section{Palabras clave}
% Incluir máximo seis (6) palabras clave que describan el objeto del
% proyecto. 

Astrof\'isica --- Cosmolog\'ia --- Materia oscura --- Energ\'ia oscura
--- Galaxias --- Computaci\'on de alto rendimiento 



\section{Planteamiento del problema}
% Delimitación clara y precisa del objeto de la investigación que
% se realiza por medio de una pregunta.

Esta propuesta busca realizar una serie de simulaciones
computacionales del Universo a gran escala. Esto lo haremos con el
objetivo de cuantificar la influencia de diferentes par\'ametros
cosmol\'ogicos sobre la formaci\'on de grandes estructuras. Esto nos
permitir\'a constribuir a colaboraciones internacionales sobre
cosmolog\'ia observacional. 

De partida nos ubicamos dentro de lo que se conoce como el modelo
es\'tandar de la cosmolog\'ia. En este modelo el contenido de materia
en el Universo est\'a dominado por la materia oscura. Adicionalemnte,
hay un componente, conocida como la constante cosmol\'ogica (la
densidad de energ\'ia asociada al espacio vac\'io), que explica la
expansi\'on acelerada del Universo. En este modelo la materia
bari\'onica es la minor\'ia en el contenido de materia ener\'gia del
Universo. La repartici\'on de estas tres componentes en t\'erminso de
fracciones de la densidad total correponden aproximadamente a un $5\%$
para los bariones, un $25\%$ para la materia oscura y un $70\%$ para
la constante cosmol\'ogica. Adicionalmente, en este modelo
consideramos que la teor\'ia que describe la interacci\'on
gravitacional es la Relatividad General de Einstein.

En este contexto, la herramienta principal para responder nuestra
pregunta original son las simulaciones num\'ericas. Los 
m\'etodos computacionales permiten simular grandes
vol\'umenes del Universo para seguir la evoluci\'on temporal de la
distribuci\'on de materia. Este ejercicio se puede hacer para
diferentes valores de los par\'ametros cosmol\'ogicos y 
medir las consecuencias de los cambios en diferentes universos
ficticios.  

El avance en los m\'etodos computacionales ha estado en gran parte
motivado por los avances en t\'ecnicas observacionales que permiten
hacer mapas del Universo a grandes escalas. Estos mapas se construyen
a partir de mediciones de los espectros de millones de
galaxias. Normalmente, estas campa\~nas observacionales son llevadas a
cabo por grandes colaboraciones internacionales y cuentan  con
cient\'ificos capaces de realizar los estudios de Universos simulados.

La presente propuesta busca entonces reponder a la pregunta sobre la
estructura del Universo a gran escala a partir de simulaciones
computacionales, teniendo en mente la aplicaci\'on del conocimiento
adquirido a datos observacionales y la planeaci\'on de futuras
observaciones.



\section{Justificaci\'on}
% Factores que hacen necesario y pertinente la realización del
% proyecto. 

Entre las razones que justifican la pertinencia de esta propuesta se
encuentran las siguientes:

\begin{itemize}
\item Contribuir a dar respuestas a preguntas actuales sobre la
  formaci\'on de estructuras en escalas cosmol\'ogicas en un Universo
  dominado por materia oscura.
\item Desarrollar m\'etodos para analizar datos observacionales de
  observaciones de distribuci\'on de galaxias a gran escala para
  inferir valores de par\'ametros comol\'ogicos.
\item Integrar colaboraciones internacionales a la frontera de la
  investigaci\'on en cosmolog\'ia internacional.
\item Generaci\'on de impacto tecnol\'ogico inmediato en el \'area de
  computaci\'on de alto rendimiento al:
\begin{itemize}
\item Realizar las simulaciones propuestas.
\item Analizar los datos de estas simulaciones.
\item Desarrollar herramientspara garantizar y analizar la
  gran cantidad de datos de las simulaciones.
\item Desarrollar e implementar nuevos algoritmos para la planeaci\'on
  de grandes campa\~nas observacionales del Universo.
\end{itemize}
\item Formaci\'on de recursos humanos en la pr\'actica de la
  computaci\'on de alto rendimiento y en el procesamiento de datos
  para hacerlos accesibles para la comunidad acad\'emica interesada. 
\end{itemize}

\section{Marco conceptual}
% Aspectos conceptuales y teóricos que contextualicen el problema de
% investigación en una temática; así como otros aspectos que sean
% pertinentes a juicio de los proponentes. 


El principal elemento conceptual de este proyecto es la {\bf Cosmolog\'ia
vista a partir de la estructura del Universo a gran escala}. Este
elemento central se divide en tres temas: teor\'ia,
simulaciones y observaciones.

\subsection{Teor\'ia}

Desde el punto de vista te\'orico hay dos observables principales que
dependen de los par\'ametros cosmol\'ogicos: la historia de
expansi\'on del Universo y el crecimiento de la estructura a gran
escala. Ambos pueden ser medidos a partir de la distribuci\'on de
galaxias en el Universo.

La {\bf historia de expansi\'on} del Universo, medida por la constante
de Hubble dependiente del redshift H(z), dependende directamente del
contenido de mater\'ia y energ\'ia del Universo. Este contenido se 
puede separar en las sigientes componentes: la densidad de materia
bari\'onica ($\Omega_b$), la densidad de materia oscura
($\Omega_{dm}$) y la densidd de energ\'ia  oscura  ($\Omega_{DE}$), la
cual puede variar en el tiempo dependiendo del valor de la constante
$w$ en la ecuaci\'on de estado, $\Omega_{DE}$ donde 

\begin{equation}
\Omega_{DE}(z) = \Omega_{DE,0}(1+z)^{3(1+w)}.
\end{equation}

Para esto hemos usado el resultado de un Universo plano con
$\Omega_b+\Omega_{dm}+\Omega_{DE}=1$ y $w$ constante en el tiempo.
Tambi\'en hemos tomado $\Omega_{DE}$ de manera general para una
Energ\'ia Oscura sin asumir que se trata de una constante
cosmol\'ogica, $\Omega_\Lambda$ con $w=-1$. 

Hoy en d\'ia estos par\'ametros cosmol\'ogicos es\'an
acotados observacionalmente cerca a los siguientes valores $\Omega_b=0.05$,
$\Omega_{dm}=0.25$, $\Omega_{DE}=0.70$ y $w=-1$. De estos
par\'ametros los que m\'as inter\'es generan actualmente son
$\Omega_{DE}$ y $w$ porque est\'an relacionados con  la expansi\'on
acelerada del Universo, algo que puede corresponder a una simple constante
cosmol\'ogica o puede ser la evidencia de que Relatividad General no
es la teor\'ia correcta para la gravedad 
\cite{2014arXiv1401.0046M}.

Por otro lado, el {\bf crecimiento de estructura a gran escala} depende
principalmente de la fluctuaciones iniciales en el campo de densidad
de materia en las \'epocas tempranas del Universo y de la teor\'ia de
la gravedad que hace que esas fluctuaciones se amplifiquen generando
la variedad de estructura que observamos en el Universo. En este
contexto la cantidad central es el contraste de densidad $\delta({\bf
  x},t)\equiv\rho({\bf r},t)/\bar{\rho(r)}-1$ de la materia oscura. En
el r\'egimen lineal de crecimiento est\'a dado por la siguiente
ecuaci\'on diferencial 

\begin{equation}
\ddot{D} + 2H(z)\dot{D}- \frac{3}{2}\Omega_mH_{0}^2(1+z)^3D=0,
\end{equation}
%
donde $H_0$ es la constante de Hubble en el presente y $D(t)$ es la
funci\'on de crecimiento. Incluso en el r\'egimen no lineal el
contraste de densidad tambi\'en depende de esta funci\'on $D(t)$.  

Lo interesante en este caso es que diferentes modelos de la gravedad
(entre ellos la Relatividad General) hacen predicciones sobre el
comportamiento de esta funci\'on $D(t)$. De esta manera una estrategia
posible para probar teorias de gravedad modificada es medir al mismo
tiempo la historia de expansi\'on y la historia de crecimiento de
estructuras para ver si dan valores consistentes de $H(z)$ y de
$w$ \cite{2014arXiv1401.0046M}. 

\subsection{Simulaciones}

Aunque la decripci\'on anal\'itica esbozada anteriorment
permite acercarse a varios fen\'omenos del crecimiento de estructura,
una descripci\'on detallada solamente ha sido posible a trav\'es de
las simulaciones computacionales. Las simulaciones buscan seguir la
evoluci\'on del campo de densidad de materia oscura en funci\'on del
tiempo. 

En el caso de simulaciones en la cosmolog\'ia est\'andar $\Lambda$
Cold Dark Matter ($\Lambda$CDM) se sigue un aproximaci\'on en la cual
solamente se simula la componente oscura y no la bari\'onica, dado que
la primera domina la din\'amica del sistema. Luego de esto se toma un
volumen computacional que represente una regi\'on del universo con
distribuci\'on de materia oscura se discrerizada en part\'iculas
computacionales con posiciones y velocidades que pueden evolucionar en
el tiempo.  Esta forma general de realizar los c\'alculos ha sido
perfeccionada durante los \'ultimos treinta a\~nos de investigaci\'on
en el tema de formaci\'on de estructura en un Universo domiando por la
materia oscura
\cite{1985ApJ...292..371D,1999ApJ...522...82K,2005Natur.435..629S}.
Aunque es posible incluir una descripci\'on del gas bari\'onico y de
los procesos asociados a las formaci\'on estelar, nosotros centraremos
nuestra discusi\'on en torno a las simulaciones que solamente incluyen
la materia oscura, dado que son suficientes para los objetivos de
nuestro proyecto.

Como en todo experimento num\'erico
hay dos cantidades centrales para la descripci\'on del sistema: las
condiciones iniciales y las reglas para la evoluci\'on temporal del
sistema.  Las condiciones iniciales est\'an definen las
posiciones y velocidades iniciales de las part\'iculas
computacionales. En este caso los desplazamientos de las part\'iculas
con respecto a una distribuci\'on homog\'ena de part\'iculas est\'an
relacionados con las fluctuaciones tempranas en el campo de densidad y
se pueden describir estad\'isiticamente a trav\'es del espectro de
potencias. Una vez las posiciones iniciales est\'an determinadas se
determinan las velocidades iniciales, usualmente a trav\'es de la
aproximaci\'on de Zeldovich aunque es posible imponer estas
condiciones iniciales a trav\'es de otras aproximaciones perturbativas
\cite{2014MNRAS.439.3630W}. 

La simulaci\'on sigue entonces la evoluci\'on de las posiciones y
velocidades de las part\'iculas durante la historia del
Universo. En el rango de evoluci\'on no lineal se forman
sobre-densidades de materia oscura que reciben el nombre de halos, en
los centros de estos halos las galaxias deberian formarse y
evolucionar. En este punto es posible entonces pasar a una
descripci\'on de la distribuci\'on de la simulaci\'on en t\'erminos de
las posiciones, velocidades y masas de estos {\bf halos de materia
  oscura}, que son la cantidades de inter\'es al momento de vincular
las predicciones de los modelos con las observaciones. Estos halos
sirven en las construcci\'on de {\bf cat\'alogos ficticios} que se
son comparables con mapas de la distribuci\'on de galaxias en el
Universo.

Para la a relizaci\'on de simulaciones  que incluyan
efectos de diferentes modelos de energ\'ia oscura hay tres
posibilidades: campos homog\'eneos de energ\'ia oscura, campos
inhomog\'eneos de energ\'ia oscura y modelos con de inhomogeneidad a
gran escala \cite{2012PDU.....1..162B}. En este proyecto nos
centraremos en los modelos $\Lambda$CDM y en los homog\'eneos de energ\'ia
oscura que se pueden incluir en simulaciones hechas con la maquinaria
$\Lambda$CDM a trav\'es de una parametrizaci\'on para la evoluci\'on
del t\'ermino $\Omega_{DE}(z)$.

\subsection{Observaciones}

Las observaciones del Universo a gran escala empezaron a tener un rol
central en la astronom\'ia observacional y en la cosmolog\'ia con dos
colaboraciones: el Sloan Digital Sky Survey (SDSS) \cite{SDSS} y el Two Degree
Field Galaxy Redshift Survey (2dFGRS) \cite{2dF}. El objetivo
principal de estas campa\~nas observacionales fue el de tomar
espectros de cerca de un mill\'on de galaxias del Universo local sobre
una gr\'an \'area del cielo. A partir de estas observaciones de la
estructura del Universo a gran escala se pueden inferir diferentes
par\'ametros cosmol\'ogicos.

Dentro de las mediciones m\'as importantes a partir de estas dos misiones
ha sido el de la densidad de materia cosmol\'ogica $\Omega_m$
\cite{2001Natur.410..169P} y el pico en la funci\'on de
correlaci\'on correspondiente a la Oscilacion Ac\'ustica de Bariones
(OAB) \cite{Eisenstein2005}. Actualmente las observaciones recientes de la OAB
son las que proveeen una de las form\'as mas competitivas de acotar la
historia de expansi\'on del Universo \cite{2014MNRAS.441...24A}.

En cuanto a las mediciones del crecimiento de estructura (cotas sobre
la forma de la funci\'on $D(t)$ mencionada en la secci\'on sobre
teor\'ia) estas se han a trav\'es de la cuantificaci\'on de la
distrbuci\'on de las posiciones de las galaxias en la direcci\'on
radial y perpendicular al observador \cite{2014MNRAS.439.3504S},
aunque los resultados son de una precisi\'on que todav\'ia no permite
distinguir claramente entre el modelo est\'andar y varias teor\'ias
modificadas de la gravedad.


\section{Estado del arte}
% Revisión actual de la temática en el contexto nacional e
% internacional, avances, desarrollos y tendencias. 

A contnuaci\'on revisamos el estado del arte de las tres tem\'aticas
planteadas en el Marco Conceptual


\subsection{Teor\'ia}

\subsection{Simulaciones}


\subsection{Observaciones}

\section{Objetivos}

\subsection*{Objetivos generales} 
% Enunciado que define de manera concreta el planteamiento del
% problema o necesidad y se inicia con un verbo en modo infinitivo, es
% medible, alcanzable y conlleva a una meta. 

\begin{itemize}
\item  Cuantificar la influencia de los par\'ametros cosmol\'ogicos en la
  estructura del Universo a gran escala.
\item Contribuir al dise\~no de un experimento de siguiente
  generaci\'on para la medici\'on de la historia de la expansi\'on del
  Universo. 
\end{itemize}

\subsection{Objetivos espe}
% Enunciados que dan cuenta de la secuencia lógica para alcanzar el
% objetivo general del proyecto. No debe confundirse con las
% actividades propuestas para dar alcance a los objetivos (ej. Tomar
% muestras en diferentes localidades de estudio); ni con el alcance de
% los productos esperados (ej. Formar un estudiante de maestría). 



\section{Metodolog\'ia}
% Exposición en forma organizada y precisa de cómo se desarrollará y
% alcanzará el objetivo general y cada uno de los objetivos
% específicos del proyecto, presentando los componentes del mismo y
% las actividades para el logro de estos. 

Para lograr los objetivos del proyecto se plane

\section{Resultados esperados de la investigaci\'on}
% Conocimiento generado en el cumplimiento de cada uno de los
% objetivos. 

\section{Resultados esperados - Productos}
% Evidencian el logro en cuanto a generación de nuevo conocimiento,
% fortalecimiento de capacidades científicas y apropiación social del
% conocimiento, incluir indicadores verificables y medibles acordes
% con los objetivos y alcance del proyecto. 

%(Productos, haciendo particular énfasis en los asociados con
%generación de nuevo conocimiento, con el fortalecimiento de
%capacidades científicas y con la apropiación social del
%conocimiento). Se deben considerar los productos relacionados en el
%documento conceptual que hace parte de la Medición de Grupos de
%Investigación, Desarrollo Tecnológico y/o Innovación, 20132 (ver
%anexo 10). 
 
\subsection{Productos resultado de actividades de Generación Nuevo
  Conocimiento} 

\subsubsection{Art\'iculos de investigaci\'on A1, A2, B y C}
% Artículos en revistas
%indexadas en los índices y bases mencionados en la Tabla I del ANEXO
%1. 

\subsection{Productos resultado de actividades de Desarrollo
  Tecnol\'ogico e Innovaci\'on} 
%Productos tecnológicos certificados o validados
%Diseño industrial, esquema de circuito integrado, software, planta
%piloto y prototipo industrial. Los requerimientos son mencionados en
%la Tabla VII del ANEXO 1. 

\subsection{Productos resultado de actividades de Apropiación Social
  del Conocimiento} 

\subsubsection{Estrategias pedag\'ogicas para el fomento de la CTI}
%Programa/Estrategia pedagógica de fomento a la CTI. Incluye la
%formación de redes de fomento de la apropiación social del
%conocimiento. Los requerimientos son mencionados en la Tabla XII del
%ANEXO 1. 

\subsubsection{Comunicaci\'onn social del conocimiento}
%Estrategias de comunicación del
%conocimiento, generación de contenidos impresos, multimedia y
%virtuales Los requerimientos son mencionados en la Tabla XIII del
%ANEXO 1. 

\subsubsection{Circulaci\'on de conocimiento especializado}

%Eventos científicos y participación en redes de conocimiento,
%documentos de trabajo (working papers), boletines divulgativos de
%resultado de investigación, ediciones de revista científica o de
%libros resultado de investigación e informes finales de
%investigación. Los requerimientos son mencionados en la Tabla XIV del
%ANEXO 1. 

\subsection{Productos de actividades relacionadas con la Formación de Recurso
Humano para la CTI }
%Tesis de Doctorado
%Dirección o co-dirección o asesoría de Tesis de Doctorado, se
%diferencian las t%esis con reconocimiento de las aprobadas. Los
%requerimientos son mencionados en la Tabla XVI del ANEXO 1. 

%Trabajo de grado de Maestría
%Dirección o co-dirección o asesoría de Trabajo de grado de maestría,
%se diferencian los trabajos con reconocimiento de los aprobados. Los
%requerimientos son mencionados en la Tabla XVI del ANEXO 1. 

%Trabajo de grado de Pregrado Dirección o co-dirección o asesoría de
%Trabajo de grao pregrado, se diferencian los trabajos con
%reconocimiento de los aprobados. Los requerimientos son mencionados
%en la Tabla XVI del ANEXO 1. 

\section{Trayectoria del equipo de investigaci\'on}
%Incluir el estado actual de investigación del equipo que conforma la
%propuesta, así como las perspectivas de investigación dentro de la
%temática enmarcada en el proyecto propuesto. 


\section{Posibles evaluadores}
%Identificar nombre y coordenadas de contacto de expertos en la
%temática de investigación a nivel nacional e internacional. 

\noindent
Octavio Valenzuela. Email {\texttt{octavio at astro unam mx}} (UNAM, M\'exico)\\
Nelson Padilla. Email {\texttt{npadilla at astro puc cl}} (PUC, Chile)\\
Patricia Tissera. Email {\texttt{patricia at iafe uba ar}} (UBA, Argentina)\\

\section{Cronograma}
%Distribución de actividades a lo largo del tiempo de ejecución del
%proyecto. Asociar a cada actividad el o los objetivos (numerados)
%relacionados con estos. 

\section{Impacto ambiental}
%Los proyectos de investigación deben incluir una reflexión
%responsable sobre los efectos positivos o negativos que puedan tener
%sobre el medio natural y la salud humana en el corto, mediano y largo
%plazo. 


En la Tabla 1 presentamos un resumen del presupuesto detallado
solicitado a Colciencias. Los items que destacan son los siguientes

\begin{itemize}

\item Apoyo para la pasant\'ia de un estudiante de doctorado en el
  Lawrence Berkeley Laboratory por un per\'iodo de seis meses.
\item Apoyo para la visita de colaboraci\'on de un asistente postdoctoral en el
  Lawrence Berkeley Laboratory por dos per\'iodos de dos meses.
\item Apoyo a viajes cortos (2 semanas) para un profesor, durante los
  tres a\~nos del proyecto. 
\item Compra de una cuchilla de procesadores y almacenamiento de 1TB
  con backup.
\end{itemize}

\begin{tabular}{|l |p{4.5cm}| c |c |c|}\hline
 & Actividad & Costo a\~no 1 & Costo a\~no 2 & Costo a\~no 3\\\hline
1. & Viaje a una escuela internacional de formaci\'on Estudiantes de
Doctorado \#1 y \#2 & & 8.000 & 8.000\\\hline
1. & Viaje a un congreso nacional para los Estudiantes de
Doctorado \#1 y \#2 & & 2.000 & 2.000 \\\hline
2. & Compra de almacenamiento. 2TB con backup & 10.000 & & \\\hline
3. & Pasant\'ia Estudiante de Doctorado \#2. (6 meses)& &  & 36.000\\\hline
4. & Viajes Investigador postdoctoral (1 meses)& & 10.000 & 10.000\\\hline
5. & Viajes internacionales profesor (1 mes) & & 10.000 & 10.000 \\ \hline
6. & Publicaciones &  & 7.000 & 7.000\\\hline 
& {\bf Total Anual} & 10.000 & 37.000 & 73.000\\\hline
& {{\bf Total Proyecto}} & \multicolumn{3}{|c|}{120.000}\\\hline
\end{tabular}


Contrapartidad Universidad de los Andes


\begin{tabular}{|l |p{4.5cm}| c |c |c|}\hline
& Actividad & Costo a\~no 1 & Costo a\~no 2 & Costo a\~no 3\\\hline
1. & Viaje a una escuela internacional de formaci\'on Estudiantes de
Doctorado \#1 y \#2 & 8.000 & & \\\hline
1. & Viaje a un congreso nacional para los Estudiantes de
Doctorado \#1 y \#2 & 2.000 & & \\\hline
5. & Viajes internacionales profesor (4 semanas) & 10.000 & & \\ \hline
6. & Publicaciones &  7.000 & & \\\hline 
2. & Salario B\'asico Jaime E. Forero-Romero (2014) & 20.000 & 20.000
& 20.000 \\\hline  

3. & Pasant\'ia Estudiante de Doctorado \#1 (1 mes) & 36.000 &
36.000 & \\\hline

4. & Salario B\'asico T\'ecnico DSIT (2014) & 5.000 & 5.000
& 5.000 \\\hline  

5. & Visita y Curso de Stefan Gottloeber (1 meses) & 8.000 & & \\\hline

& {\bf Total Anual} & 96.000 & 61.000 & 25.000\\\hline
& {{\bf Total Proyecto}} & \multicolumn{3}{|c|}{185.000}\\\hline

\end{tabular} 

La dedicaci\'on al proyecto es de 15\% del tiempo para
profesores. El costo mensual del salario se compone de un valor
b\'asico (X.XXX) m\'as un factor prestacional (X.XXX) Se asumen
incrementos salariales anuales del 4\%. 

La dedicaci\'on al proyecto es de 15\% para los t\'ecnicos encargados
de mantener e instalar los recursos de computaci\'on de alto
rendimiento. 


\section{Bibliograf\'ia}
%Fuentes bibliográficas empleadas en cada uno de los ítems del
%proyecto. Se hará referencia únicamente a aquellas fuentes empleadas
%en el suministro de la información del respectivo proyecto. No se
%incluirán referencias que no se citen. Las citas, en cada uno de los
%campos del formulario, se harán empleando el número de la referencia
%dentro de paréntesis cuadrados (p. ej. [1]). 

\bibliography{referencias}{}
\bibliographystyle{plain}


\end{document}
