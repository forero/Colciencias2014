\section{Objetivos}

\subsection{Objetivos generales} 
% Enunciado que define de manera concreta el planteamiento del
% problema o necesidad y se inicia con un verbo en modo infinitivo, es
% medible, alcanzable y conlleva a una meta. 

Cuantificar la influencia de los par\'ametros cosmol\'ogicos en la
  estructura del Universo a gran escala con herramientas y m\'etodos competitivos y \'utiles para la comunidad internacional en Cosmolog\'ia. Para ello nos proponemos:

\begin{enumerate}
\item  Realizar simulaciones de formarci\'on de estructura a gran escala en el Universo.
\item Utilizar datos observacionales de la distribuci\'on de galaxias a gran escala para acotar valores de par\'ametros cosmol\'ogicos.
\item Contribuir al dise\~no de DESI (Dark Energy Spectroscipic Instrumet) un experimento de siguiente generaci\'on para la medici\'on de la historia de la expansi\'on del Universo. 
\end{enumerate}

Los datos y m\'etodos 

\subsection{Objetivos espec\'ificos}
% Enunciados que dan cuenta de la secuencia lógica para alcanzar el
% objetivo general del proyecto. No debe confundirse con las
% actividades propuestas para dar alcance a los objetivos (ej. Tomar
% muestras en diferentes localidades de estudio); ni con el alcance de
% los productos esperados (ej. Formar un estudiante de maestría). 


\begin{enumerate}
\item Desarrollas en colombia la Infraestructura computacional necesarioa para hacer simulaciones cosmol\'ogicas, esto incluye las componentes de software y de ardware apropiadas.
\end{enumerate}
