\section{Objetivos}

\subsection{Objetivos generales} 
% Enunciado que define de manera concreta el planteamiento del
% problema o necesidad y se inicia con un verbo en modo infinitivo, es
% medible, alcanzable y conlleva a una meta. 

Cuantificar la influencia de los par\'ametros cosmol\'ogicos en la
  estructura del Universo a gran escala con herramientas y m\'etodos competitivos y \'utiles para la comunidad internacional en Cosmolog\'ia. Para ello nos proponemos:

\begin{enumerate}
\item  Realizar simulaciones de formarci\'on de estructura a gran escala en el Universo.
\item Contribuir al dise\~no de DESI (Dark Energy Spectroscipic Instrumet) un experimento de siguiente generaci\'on para la medici\'on de la historia de la expansi\'on del Universo. 
\item Utilizar datos observacionales de la distribuci\'on de galaxias a gran escala para acotar valores de par\'ametros cosmol\'ogicos.

\end{enumerate}

El software generado por el proyecto {\bf CoCO} estar\'a a disponibilidad  de la comunidad astron\'omica internacional a trav\'es de repositorios p\'ublicos. En el caso de los datos se conservar\'an en discos duros con backup y estar\'an disponibles para la comunidad astron\'omica internacional bajo solicitud. 

\subsection{Objetivos espec\'ificos}
% Enunciados que dan cuenta de la secuencia lógica para alcanzar el
% objetivo general del proyecto. No debe confundirse con las
% actividades propuestas para dar alcance a los objetivos (ej. Tomar
% muestras en diferentes localidades de estudio); ni con el alcance de
% los productos esperados (ej. Formar un estudiante de maestría). 


\begin{enumerate}
\item Desarrollar la Infraestructura computacional necesaria para hacer simulaciones cosmol\'ogicas; incluyendo la adaptaci\'on o desarrollo de componentes de software necesarias y la compra del hardware apropiado. Para esto prevemos la compra de 6TB de memoria con backup y un blade de 24 procesadores y 512GB de RAM para poder tener capacidad de c\'omputo dedicada exclusivamente a \coco.
\item Hacer y analizar simulaciones de formaci\'on de estructura a gran escala en Universos dominados por materia oscura en una cosmolog\'ia LCDM. Estas simulaciones tendr\'an $1000$ \hMpc de lado con $1024^3$ part\'iculas. Por cada simulaci\'on se guardar\'an en disco $5$ snapshots con las posiciones y velocidades de todas las part\'iculas. Esto representa cerca de $300$ GB en espacio de disco y cerca de 15000 horas de CPU (cerca de 10 d\'ias sobre 64 procesadores). Esta simulaci\'on utilizar\'ia cerca de 120GB de RAM al momento de correr. Estas condiciones son realizables en el cluster de la Universidad de los Andes, el cual tiene cerca de $500$ procesadores y cerca de $1$TB de RAM distribuida.
\item Hacer y analizar simulaciones de formaci\'on de estructura a gran escala en Universos dominados por materia oscura en cosmolog\'ias $w$CDM con las mismas condiciones t\'ecnicas de las simulaciones LCDM.
\item Construir cat\'alogos ficticios de galaxias; incluyendo la adpataci\'on o desarrollo de componentes de sofware necesarias.
\item Colaborar en el dise\~no del experimento DESI, probando diferentes Secuencias de observaci\'on del cielo; 
incluyendo la adaptaci\'on o desarrollo de compoentes de software necesarias.
\item Colaborar en el dise\~no del experimento DESI, probando diferentes algoritmos de distribuci\'o de fibras \'opticas sobre galaxias; incluyendo la adaptaci\'on o desarrollo de compoentes de software necesarias.
\item Colaborar en el dise\~no del experimento DESI, ayudando en la integraci\'on del sofware que simula el experimento de manera completa, desde las observaciones hasta la determinaci\'on de los redshifts de las galaxias observadas.
\item Cuantificar el efecto Alcock-Paczinsky sobre el beta-skeleton construido sobre la distribuci\'on tridimensional de galaxias.
\item Cuantificar las estad\'isiticas de pares de velocidades de encuentros extremos de pares de c\'umuloes de galaxias en cosmolog\'ias LCDM y $w$CDM.
\item Cuantificar el efecto de Redshift Space Distortions sobre las estad\'isticas de clustering de galaxias. 
\end{enumerate}


