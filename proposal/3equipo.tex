\section{Conformaci\'on del equipo de investigaci\'onn}
%Colocar el nombre y código, registrado en el GrupLac, del o de los
%grupos de investigación. Al igual que el nombre de los demás
%integrantes que conforman el equipo de trabajo. Se debe incluir el
%tiempo de dedicación y funciones en el marco del proyecto.  

La investigaci\'on se har\'a dentro del grupo de astrof\'isica de la
Universidad de los Andes, c\'odigo GrupLAC COL0015473. Los integrantes
del equipo son los siguientes. 


\begin{itemize}
\item Un profesor de planta del grupo de Astrof\'isica de la
  Universidad de los Andes: Jaime Ernesto  Forero Romero, PhD.
  Dedicaci\'on: 7 horas/semana.
\item Dos estudiantes de doctorado en el departamento de F\'isica de
  la Universidad de los Andes.
\begin{itemize}
\item Felipe Leonardo G\'omez Cort\'es
  (F\'isico) con participaci\'on activa en los 36 meses del proyecto. Dedicaci\'on: 20 horas / semana.  
  \item Estudiante por definir con participaci\'on activa en los 24 \'ultimos
    meses del proyecto (los primeros 12 estar\'an enfocados a
    actividades formativas).  Dedicaci\'on: 20 horas / semana.  
\end{itemize}
\end{itemize}

\noindent
Con el apoyo de los siguientes asesores internacionales:

\begin{itemize}

\item Stefan Gottloeber, PhD. Cient\'ifico en el Leibniz Institute for
  Astrophysics, Alemania.  
\item Changbom Park, PhD. Miembro permanente del Korean Institute for
  Advanced Studies, Corea del Sur. 
\item Robert Cahn, PhD. Cient\'ifico en el Lawrence Berkeley National
  Laboratory, Estados   Unidos. 
\end{itemize}

\noindent
Adicionalmente el proyecto cuenta con el soporte del siguiente
personal t\'ecnico de la Universidad de los Andes:

\begin{itemize}
\item{Ingeniero de c\'omputo del departamento de F\'isica.}
\item{Ingeniero de c\'omputo de alto rendimiento en el Direcci\'on de
  Servicios de Informaci\'on y Tecnolog\'ia.}
\end{itemize}
