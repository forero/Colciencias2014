

\section{Planteamiento del problema}
% Delimitación clara y precisa del objeto de la investigación que
% se realiza por medio de una pregunta.

Esta propuesta busca realizar una serie de simulaciones
computacionales del Universo a gran escala. Esto lo haremos con el
objetivo de cuantificar la influencia de diferentes par\'ametros
cosmol\'ogicos sobre la formaci\'on de grandes estructuras. Esto nos
permitir\'a constribuir a colaboraciones internacionales sobre
cosmolog\'ia observacional. 

De partida nos ubicamos dentro de lo que se conoce como el modelo
es\'tandar de la cosmolog\'ia. En este modelo el contenido de materia
en el Universo est\'a dominado por la materia oscura. Adicionalemnte,
hay un componente, conocida como la constante cosmol\'ogica (la
densidad de energ\'ia asociada al espacio vac\'io), que explica la
expansi\'on acelerada del Universo. En este modelo la materia
bari\'onica es la minor\'ia en el contenido de materia ener\'gia del
Universo. La repartici\'on de estas tres componentes en t\'erminso de
fracciones de la densidad total correponden aproximadamente a un $5\%$
para los bariones, un $25\%$ para la materia oscura y un $70\%$ para
la constante cosmol\'ogica. Adicionalmente, en este modelo
consideramos que la teor\'ia que describe la interacci\'on
gravitacional es la Relatividad General de Einstein.

En este contexto, la herramienta principal para responder nuestra
pregunta original son las simulaciones num\'ericas. Los 
m\'etodos computacionales permiten simular grandes
vol\'umenes del Universo para seguir la evoluci\'on temporal de la
distribuci\'on de materia. Este ejercicio se puede hacer para
diferentes valores de los par\'ametros cosmol\'ogicos y 
medir las consecuencias de los cambios en diferentes universos
ficticios.  

El avance en los m\'etodos computacionales ha estado en gran parte
motivado por los avances en t\'ecnicas observacionales que permiten
hacer mapas del Universo a grandes escalas. Estos mapas se construyen
a partir de mediciones de los espectros de millones de
galaxias. Normalmente, estas campa\~nas observacionales son llevadas a
cabo por grandes colaboraciones internacionales y cuentan  con
cient\'ificos capaces de realizar los estudios de Universos simulados.

La presente propuesta busca entonces reponder a la pregunta sobre la
estructura del Universo a gran escala a partir de simulaciones
computacionales, teniendo en mente la aplicaci\'on del conocimiento
adquirido a datos observacionales y la planeaci\'on de futuras
observaciones.
