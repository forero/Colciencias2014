

\section{Planteamiento del problema}
% Delimitación clara y precisa del objeto de la investigación que
% se realiza por medio de una pregunta.

La pregunta que dirige la investigaci\'on de este proyecto es: ¿Cómo
se estructura el Universo en el que vivimos?

Nuestro objetivo principal es cuantificar la influencia de diferentes
par\'ametros cosmol\'ogicos sobre la formaci\'on estructura en el Universo.
Para esto haremos simulaciones de la evolución de estructuras en
grandes escalas lo que nos permitirá poner pie en una colaboración
internacional que busca observar lo que hemos simulado en la
computadora.

Para entender esta perspectiva de trabajo es necesario ubicarnos en el
el modelo es\'tandar de la cosmolog\'ia. 
En este modelo el contenido de materia en el Universo est\'a dominado
por la materia oscura. 
Adicionalemnte, hay uns componente, conocida como la constante cosmol\'ogica (la
densidad de energ\'ia asociada al espacio vac\'io), que explica la
expansi\'on acelerada del Universo. 
En este modelo la materia bari\'onica es la minor\'ia en el contenido
de materia ener\'gia del Universo. 
La repartici\'on de estas tres componentes en t\'erminos de fracciones
de la densidad total correponden aproximadamente a un $5\%$ para los
bariones, un $25\%$ para la materia oscura y un $70\%$ para la
constante cosmol\'ogica. 
Adicionalmente, en este modelo consideramos que la teor\'ia que
describe la interacci\'on gravitacional es la Relatividad General de
Einstein. 

En este contexto, {\bf las simulaciones computacionales son la mejor
herramienta} para cuantificar los efectos de diferentes par\'ametros
cosmol\'ogicos en la formaci\'on de estructuras. 
Los  m\'etodos computacionales permiten simular grandes vol\'umenes
del Universo para seguir la evoluci\'on temporal de la distribuci\'on
de materia. 
Este ejercicio se puede hacer para diferentes valores de los
par\'ametros cosmol\'ogicos y  medir las consecuencias de los cambios
en diferentes universos ficticios.  

El gran avance en los m\'etodos computacionales ha estado motivado por
los avances en t\'ecnicas observacionales que permiten hacer mapas del
Universo a grandes escalas a partir de mediciones de los espectros de
millones de galaxias.   
Actualmente, estas campa\~nas observacionales tambi\'en se
retroalimentan de los resultados de las simulaciones. 
Esto hace que finalmente las simulaciones computacionales sean una
herramienta \'util para los te\'oricos y los observacionales. {\bf Sin el
trabajo conjunto de simulaciones y observaciones ser\'ia imposible
inferir la estructura que tiene el Universo.}

En \'ultimas, nuestra propuesta dar respuesta a la pregunta sobre la
estructura del Universo realizando simulaciones para estudiar posibles
efectos medibles y as\'i mismo aplicar este conocimiento en la
planeaci\'on de futuras observaciones astron\'omicas lideradas por
colabraciones internacionales. 
