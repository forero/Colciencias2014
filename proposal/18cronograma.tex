\section{Cronograma}
%Distribución de actividades a lo largo del tiempo de ejecución del
%proyecto. Asociar a cada actividad el o los objetivos (numerados)
%relacionados con estos. 

Los responsables de las actividades ser\'an:

\begin{itemize}
\item Dr. Jaime E. Forero Romero  (PROF)
\item Felipe G\'omez, estudiante de doctorado (GRAD1)
\item Estudiante de doctorado por determinar (GRAD2)
\item Asistentes t\'ecnicos de Uniandes (TECN)
\end{itemize}

El cronograma de actividades es\'a dividio por semestres

\begin{itemize}
\item[\bf S1]
\begin{itemize}
\item {\bf (TECN)} Compra e instalaci\'on de 3TB de memoria con backup
  para almacenar las simulaciones.
\item {\bf (TECN)} Compra e instalaci\'on de 1 Blade con 24
  procesadores y 512GB de RAM.
\item {\bf (GRAD1)} Correr la primera serie de 5 simulaciones en la
  cosmolog\'ia LCDM. 
\item {\bf (GRAD1)} Procesar esta serie de 5 simulaciones para
  identificar los halos de materia oscura.
\item {\bf (PROF)} Desarrollar herramientas b\'asicas para hacer cat\'alogos
  ficticios aleatorios para DESI.  
\item {\bf (PROF)} Desarrollar c\'odigo para la asignaci\'on de fibras
  \'opticas y cronogramas de observaci\'on en galaxias de catalogos
  ficticios aleatorios. 
\item {\bf (PROF)} Visita de 2 semanas a Berkeley para Integraci\'on del
  c\'odigo desarrollado en la base general de c\'odigo de DESI.
\item {\bf (PROF - GRAD1 - GRAD2)} Organizar la escuela Internacional de
  Comoslog\'ia Computacional.   
\end{itemize}


\item[{\bf S2}]
\begin{itemize}
\item {\bf (GRAD1)} Desarrollar y adaptar herramientas b\'asicas para construir.
  cat\'alogos ficticios a partir de simulaciones.
\item {\bf (GRAD2)} An\'alisis de cat\'alogos ficticios para medir
  el efecto Alcock-Paczinsky en el beta-skeleton.
\item {\bf (GRAD1)} Visita a escuela internacional de cosmolog\'ia computacional/observacional.
\item {\bf (GRAD2)} Visita a escuela internacional de cosmolog\'ia computacional/observacional.
\item {\bf (PROF)} Refinar algoritmos de asignaci\'on de fibras
  \'opticas y cronogramas de observaci\'on en galaxias de cat\'alogos
  ficticios generados a partir de simulaciones.
\item {\bf (PROF)} Visita de 2 semanas a Seoul para desarrollar c\'odigo que 
  extienda la medici\'on del efecto Alcock-Paczinsky a observaciones
  de galaxias de SDSS-III (requiere la interacci\'on directa con
  Chamgbom Park y su equipo).
\end{itemize}


\item[\bf S3]
\begin{itemize}
\item {\bf (TECN)} Compra e instalaci\'on de 3TB de memoria con backup
  para almacenar las simulaciones. 
\item {\bf (GRAD1)} Correr la segunda serie de 5 simulaciones en 
  cosmolog\'ias $w$CDM.  
\item {\bf (GRAD1)} Procesar esta serie de 5 simulaciones $w$CDM para
  identificar los halos de materia oscura. 
\item {\bf (GRAD2)} Hacer el an\'alisis
  del Alcock-Paczinski en el beta-skeleton sobre observaciones de
  SDSS-III. 
\item {\bf (PROF)} Terminar de implementar el c\'odigo que simula el
  cronograma de observaci\'on, la asignaci\'on de fibras sobre
  cat\'alogos ficticios generados a partir de simulaciones.
\item {\bf (PROF)} Visita de 2 semanas a Berkeley para Integraci\'on del
  c\'odigo desarrollado en este semestre a la base general de c\'odigo
  de DESI. 
\end{itemize}

\item[\bf S4]
\begin{itemize}
\item {\bf (GRAD1)} Analizar en las simulaciones la influencia de las diferentes
  cosmolog\'ias en la distribuci\'on de pares de velocidades de
  c\'umulos de galaxias.
\item {\bf (GRAD1)} Viaje a Seoul para usar simulaciones de KIAS extender el estudio de  influence de diferentes cosmolog\'ias en pares de velocidades.
\item {\bf (GRAD2)} Analizar la influencia de las diferentes
  cosmolog\'ias en las Redshift Space Distorsions de cat\'alogos
  creados a partir de simulaciones.
\item {\bf (GRAD2)} Viaje a Berkeley por 2 semanas para terminar el
  estudio sobre la influenci de RSD en cat\'alogos simulados.
\item {\bf (PROF)} Generar cat\'alogos ficiticios de galaxias a partir
  de simulaciones para diferentes grupos de trabajo en DESI.
\end{itemize}


\item[\bf S5]
\begin{itemize}
\item {\bf (GRAD1)} Viaje a Berkeley para interactuar con los investigadores
  principales de diferentes working groups de DESI para definir sus necesidades
  de cat\'alogos ficticios y aprender sobre algoritmos para medici\'on
  de clusterins anisotr\'opico.
\item {\bf (GRAD1)} Generar cat\'alogos ficticios de galaxias a partir
  de simulaciones para diferentes grupos de trabajo en DESI. 
\item {\bf (GRAD2)} Medir la posible influencia/utilidad de RSD 
sobre la estimaci\'on de par\'ametros utilizando cat\'alogos ficticios
construidos a partir cosmol\'ogicos. 
\item {\bf (GRAD2)} Incluye una vista a Seoul para trabajar sobre
  datos observacionales de SDSS-III y entrenarse en el uso de datos
  observacionales para medir estad\'isticas de RSD.
\item {\bf (PROF)} Visita de dos semanas a Berkeley para colaborar en
  la integraci\'on completa del c\'odigo    que simula DESI desde las
  observaciones hasta la estimaci\'on de  par\'ametros cosmol\'ogicos.  
\end{itemize}

\item[\bf S6]
\begin{itemize}
\item {\bf (PROF)} Correr el c\'odigo completo que simula DESI desde
  las observaciones hasta la estimaci\'on de par\'ametros
  cosmol\'ogicos. 
\item {\bf (GRAD1)} Medir el clustering anisotr\'opico sobre los datos
  de la simulaci\'on completa de DESI.
\item {\bf (GRAD2)} Medir la influencia/utilidad de RSD sobre la
  estimaci\'on de par\'ametros cosmol\'ogicos a partir de los
  resultados de la simulaci\'on completa de DESI.

\end{itemize}

\end{itemize}
