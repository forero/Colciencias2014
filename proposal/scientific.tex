\documentclass[12pt]{article}
\usepackage[spanish]{babel}
\usepackage[utf8]{inputenc}
\newcommand{\apj}{ApJ}
\newcommand{\apjs}{ApJS}
\newcommand{\apjl}{ApJL}
\newcommand{\aj}{AJ}
\newcommand{\mnras}{MNRAS}
\newcommand{\mnrassub}{MNRAS accepted}
\newcommand{\aap}{A\&A}
\newcommand{\aaps}{A\&AS}
\newcommand{\araa}{ARA\&A}
\newcommand{\nat}{Nature}
\newcommand{\prd}{PRD}
\newcommand{\physrep}{PhR}
\newcommand{\pasp}{PASP}
\newcommand{\pasj}{PASJ} 
\title{Proyecto:\\{\bf Cosmolog\'ia Computacional y Observacional}}
\author{Investigador Principal:\\{\bf Jaime Ernesto Forero Romero,
    PhD}\\
}
\begin{document}
\maketitle
\tableofcontents 
\newpage

\section{T\'itulo del proyecto}
\noindent
Cosmolog\'ia Computacional y Observacional

\section{Investigador principal}

\noindent
Jaime Ernesto Forero Romero, PhD. Profesor asistente Universidad de los Andes. 


\section{Conformaci\'on del equipo de investigaci\'onn}
%Colocar el nombre y código, registrado en el GrupLac, del o de los
%grupos de investigación. Al igual que el nombre de los demás
%integrantes que conforman el equipo de trabajo. Se debe incluir el
%tiempo de dedicación y funciones en el marco del proyecto.  

\begin{itemize}
\item Un profesor de planta del grupo de Astrof\'isica de la
  Universidad de los Andes: Jaime Ernesto  Forero Romero.  
\item Dos estudiantes de doctorado en el departamento de F\'isica de
  la Universidad de los Andes: Felipe G\'omez (F\'isico) y otro por
  definir.  
\end{itemize}

\noindent
Con el apoyo de los siguientes asesores internacionales:

\begin{itemize}

\item Stefan Gottloeber, PhD. Cient\'ifico en el Leibniz Institute for Astrophysics, Alemania.
\item Changbom Park, PhD. Miembro permanente del Korean Institute for Advanced Studies, Corea del Sur.
\item Robert Cahn, PhD. Cient\'ifico en el Lawrence Berkeley National Laboratory, Estados   Unidos.
\end{itemize}

\noindent
Adicionalmente el proyecto cuenta con el soporte del siguiente
personal de apoyo de la Universidad de los Andes 

\begin{itemize}
\item{Ingeniero de c\'omputo del departamento de F\'isica.}
\item{Ingeniero de c\'omputo de alto rendimiento en el Departamento de Servicios de Tecnolog\'ia de la Informaci\'on}.
\end{itemize}

\section{Antecedentes y resultados previos del equipo de
  investigaci\'on} 
%  investigación solicitante en la temática específica del proyecto} 
%trayectoria del equipo de investigación con relación al problema
%planteado en el proyecto. 

\section{Tem\'atica de investigaci\'on}
% Especificar en cuál de las temáticas definidas por la convocatoria
% está enmarcado el proyecto. 


\section{Resumen ejecutivo}
% Información mínima necesaria para comunicar de manera precisa los
% contenidos y alcances del proyecto. 

La cosmología observacional entró en una época dorada con la medición
de las anisotropías de la radiación cósmica de fondo (Premio Nobel de
Física 2006) y la medición de la expansión acelerada del Universo
(Premio Nobel de Física 2011). 

Hoy en d\'ia una de las fronteras de la investigación en {\bf Cosmología
Observacional} es la obtenci\'on de mejores mediciones de la historia de
expansión acelerada del Universo y as\'i inferir cotas sobre uno de
los posibles responsables de este efecto: la Energ\'ia Oscura.  Una de
las t\'ecnicas observacionales que se utiliza con ese prop\'osito es
la detecci\'on del pico de Oscilaciones Ac\'usticas de
Bariones (OAB) que requiere la medici\'on de las \emph{posiciones} de
millones de galaxias \cite{Eisenstein2005}. Otras m\'etodos usan
informaci\'on sobre las \emph{velocidades} peculiares (i.e. velocidades
que no es\'an asociadas al flujo de Hubble) de las galaxias y las distorsiones
que estas generan en las observaciones para cuantificar el crecimiento
de estructura a gran escala \cite{Scoccimarro2004}.

En todos estos esfuerzos observacionales para medir el Universo las
{\bf simulaciones computacionales} juegan un rol fundamental, especialmente
en los m\'etodos que usan las velocidades peculiares. Las simulaciones
son utilizadas para traducir las premisas te\'oricas en cantidades
observables. Es decir, son un puente entre la teor\'ia y la
observaci\'on. 

Las simulaciones de formaci\'on de estructura a gran escala en el
Universo han llegado a un gran punto de madurez durante los \'ultimos
treinta a\~nos de investigaci\'on en el tema dando nacimiento a un
\'area de investigaci\'on conocida como {\bf Cosmolog\'ia
  Computacional}. Hoy en d\'ia se utilizan las mismas simulaciones al
momento de planear una nueva campa\~na observacional; el fin principal
es asegurarse de que los efectos f\'isicos que se esperan medir
est\'an al alcance del instrumento y de los m\'etodos estad\'isticos
para analizar los datos. 

La presente propuesta tiene como objetivo hacer uso de los fondos para
hacer investigaci\'on en Cosmolog\'ia Computacional y Cosmolog\'ia
Observacional.  

En el frente de
cosmolog\'ia computacional proponemos realizar una serie de
simulaciones de la distribuci\'on de materia en el Universo a gran
escala sin precedentes en el pa\'is. Para la preparaci\'on de estas
simulaciones haremos uso de las m\'aquinas de las Instalaciones para
C\'omputo de Alto Rendimiento de la Universidad de los Andes, las
cuales ser\'an instaladas y puestas en funcionamiento durante el
segundo semestre del 2014.

Vamos a usar estas simulaciones de manera principal para fundamentar
nuestro trabajo en Cosmolog\'ia Observacional en dos colaboraciones
internacionales. 

La primera colaboraci\'on, con el Lawrence Berkeley National Laboratory,
se trata de contribuir en la preparaci\'on del experimento DESI (Dark
Energy Spectroscopic Instrument). DESI es una colaboraci\'on internacional de
m\'as de 100 cient\'ificos que construir\'a el experimento de
siguiente generaci\'on para medir la historia de expansi\'on del
Universo haciendo un mapa de la distribuci\'on de 25 millones de
galaxias, 10 veces m\'as de lo que ha sido observado hasta la fecha
\cite{DESI2013}. En particular, esperamos durante la duraci\'on del
presente proyecto con COLCIENCIAS aportar al dise\~no y optimizaci\'on
de la estrategia de las observaciones. 

La segunda colaboraci\'on, con el Korean Insitute for Advanced Studies
(KIAS), trata de la la restricci\'on de par\'ametros cosmol\'ogicos a
partir de la anisotrop\'ia de la distribuci\'on de galaxias. En una
primera etapa esperamos usar las simulaciones para optimizar los
m\'etodos estad\'isticos de medici\'on del efecto buscado. En una
segunda etapa vamos a aplicar este conocimiento sobre los datos
observacionales del proyecto SDSS-III (Sloan Digital Sky Survey),
algo posible gracias a la participaci\'on de KIAS en ese proyecto.

Los proyectos posibles con la serie de simulaciones que vamos a
realizar abren la puerta a otro tipo de estudios sobre la estructura
del Universo a gran escala \cite{Tweb,Vweb} y la cuantificaci\'on de
la abundance eventos de colisiones extremas en el Universo
\cite{Bullets2010,Bullets2014}. Adicionalmente, estos datos se
har\'an disponibles a toda la comunidad (colombiana e internacional)
para realizar diferentes tipos de estudios y an\'alisis \cite{Multidark}. 
En el curso de este proyecto tambi\'en vamos a organizar una escuela
internacional de un mes, con expertos internacional, para formar 
recurso humano en Colombia y en la regi\'on que est\'e en capacidad de
utilizar recursos computacionales avanzados para resolver problemas en
cosmolog\'ia computacional y observacional.



\section{Palabras clave}
% Incluir máximo seis (6) palabras clave que describan el objeto del
% proyecto. 

Astrof\'isica --- Cosmolog\'ia --- Materia oscura --- Energ\'ia oscura
--- Galaxias --- Computaci\'on de alto rendimiento 



\section{Planteamiento del problema}
% Delimitación clara y precisa del objeto de la investigación que
% se realiza por medio de una pregunta.

La pregunta principal que queremos responder es la siguiente: ¿cómo se
distribuye la materia en el Universo a gran escala? M\'as precisamente,
queremos cuantificar los cambios en esta distribuci\'on en funci\'on
del tiempo y de los par\'ametros cosmol\'ogicos que describen el
Universo.

De partida nos ubicamos dentro de lo que se conoce como el modelo
es\'tandar de la cosmolog\'ia. En este modelo el contenido de materia
en el Universo est\'a dominado por la materia oscura. Adicionalemnte,
hay un componente, conocida como la constante cosmol\'ogica (la
densidad de energ\'ia asociada al espacio vac\'io), que explica la
expansi\'on acelerada del Universo. En este modelo la materia
bari\'onica es la minor\'ia en el contenido de materia ener\'gia del
Universo. La repartici\'on de estas tres componentes en t\'erminso de
fracciones de la densidad total correponden aproximadamente a un $5\%$
para los bariones, un $25\%$ para la materia oscura y un $70\%$ para
la constante cosmol\'ogica. Adicionalmente, en este modelo
consideramos que la teor\'ia que describe la interacci\'on
gravitacional es la Relatividad General de Einstein.

En este contexto, la herramienta principal para responder nuestra
pregunta original son las simulaciones num\'ericas. Los 
m\'etodos computacionales permiten simular grandes
vol\'umenes del Universo para seguir la evoluci\'on temporal de la
distribuci\'on de materia. Este ejercicio se puede hacer para
diferentes valores de los par\'ametros cosmol\'ogicos y 
medir las consecuencias de los cambios en diferentes universos
ficticios.  

El avance en los m\'etodos computacionales ha estado en gran parte
motivado por los avances en t\'ecnicas observacionales que permiten
hacer mapas del Universo a grandes escalas. Estos mapas se construyen
a partir de mediciones de los espectros de millones de
galaxias. Normalmente, estas campa\~nas observacionales son llevadas a
cabo por grandes colaboraciones internacionales y cuentan  con
cient\'ificos capaces de realizar los estudios de Universos simulados.

La presente propuesta busca entonces reponder a la pregunta sobre la
estructura del Universo a gran escala a partir de simulaciones
computacionales, teniendo en mente la aplicaci\'on del conocimiento
adquirido a datos observacionales y la planeaci\'on de futuras
observaciones.

\section{Justificaci\'on}
% Factores que hacen necesario y pertinente la realización del
% proyecto. 

Los principales factores que hacen necesaria y pertinente la
realizaci\'on de este proyecto.

\section{Marco conceptual}
% Aspectos conceptuales y teóricos que contextualicen el problema de
% investigación en una temática; así como otros aspectos que sean
% pertinentes a juicio de los proponentes. 


El principal elemento conceptual de este proyecto es la {\bf Cosmolog\'ia
vista a partir de la estructura del Universo a gran escala}. Este
elemento central se divide en tres temas: teor\'ia,
simulaciones y observaciones.

\subsection{Teor\'ia}

Desde el punto de vista te\'orico hay dos observables principales que
dependen de los par\'ametros cosmol\'ogicos: la historia de
expansi\'on del Universo y el crecimiento de la estructura a gran
escala. Ambos pueden ser medidos a partir de la distribuci\'on de
galaxias en el Universo.

La {\bf historia de expansi\'on} del Universo dependende directamente del
contenido de mater\'ia y energ\'ia del Universo. Este contenido se
puede separar en las sigientes componentes: la densidad de materia
bari\'onica ($\Omega_b$), la densidad de materia oscura
($\Omega_{dm}$) y la densidd de energ\'ia asociada a la constante
cosmol\'ogica ($\Omega_{\Lambda}$), la cual puede variar en el tiempo
dependiendo del valor de la constante $w$ en la ecuaci\'on de
estado, donde hemos usado el resultado de un Universo plano con $\Omega_b+\Omega_{dm}+\Omega_{\Lambda}=1$.

Hoy en d\'ia estos par\'ametros cosmol\'ogicos es\'an
acotados observacionalmente cerca a los siguientes valores $\Omega_b=0.05$,
$\Omega_{dm}=0.25$, $\Omega_{\Lambda}=0.70$ y $w=-1$. De estos
par\'ametros los que m\'as inter\'es generan actualmente son
$\Omega_\Lambda$ y $w$ porque est\'an relacionados con la misteriosa
energ\'ia oscura que explicar\'ia la expansi\'on acelerada del
Universo, algo que puede corresponder a una simple constante
cosmol\'ogica o puede ser la evidencia de que Relatividad General no
es la teor\'ia correcta para la gravedad 
\cite{2014arXiv1401.0046M}.

El {\bf crecimiento de estructura a gran escala} depende
principalmente de la fluctuaciones iniciales en el campo de densidad
de materia en las \'epocas tempranas del Universo y de la teor\'ia de
la gravedad que hace que esas fluctuaciones se amplifiquen generando
la variedad de estructura que observamos en el Universo.




\subsection{Simulaciones}


\subsection{Observaciones}

Las observaciones del Universo a gran escala empezaron a tener un rol
central en la astronom\'ia observacional y en la cosmolog\'ia con dos
colaboraciones: el Sloan Digital Sky Survey (SDSS) \cite{SDSS} y el Two Degree
Field Galaxy Redshift Survey (2dFGRS) \cite{2dF}. El objetivo
principal de estas campa\~nas observacionales fue el de tomar
espectros de cerca de un mill\'on de galaxias del Universo local sobre
una gr\'an \'area del cielo. A partir de estas observaciones de la
estructura del Universo a gran escala se pueden inferir diferentes
par\'ametros cosmol\'ogicos.

Dentro de las mediciones m\'as importantes a partir de estas dos misiones
ha sido el de la densidad de materia cosmol\'ogica $\Omega_m$
\cite{2001Natur.410..169P} y el pico en la funci\'on de
correlaci\'on correspondiente a la Oscilacion Ac\'ustica de Bariones
(OAB) \cite{Eisenstein2005}.  







 


\section{Estado del arte}
% Revisión actual de la temática en el contexto nacional e
% internacional, avances, desarrollos y tendencias. 

A contnuaci\'on revisamos el estado del arte de las tres tem\'aticas
planteadas en el Marco Conceptual


\subsection{Teor\'ia}

\subsection{Simulaciones}

\subsection{Observaciones}

\section{Objetivos}

\subsection*{Objetivos generales} 
% Enunciado que define de manera concreta el planteamiento del
% problema o necesidad y se inicia con un verbo en modo infinitivo, es
% medible, alcanzable y conlleva a una meta. 

\begin{itemize}
\item  Cuantificar la influencia de los par\'ametros cosmol\'ogicos en la
  estructura del Universo a gran escala.
\item Contribuir al dise\~no de un experimento de siguiente
  generaci\'on para la medici\'on de la historia de la expansi\'on del
  Universo. 
\end{itemize}

\subsection{Objetivos espe}
% Enunciados que dan cuenta de la secuencia lógica para alcanzar el
% objetivo general del proyecto. No debe confundirse con las
% actividades propuestas para dar alcance a los objetivos (ej. Tomar
% muestras en diferentes localidades de estudio); ni con el alcance de
% los productos esperados (ej. Formar un estudiante de maestría). 



\section{Metodolog\'ia}
% Exposición en forma organizada y precisa de cómo se desarrollará y
% alcanzará el objetivo general y cada uno de los objetivos
% específicos del proyecto, presentando los componentes del mismo y
% las actividades para el logro de estos. 

\section{Resultados esperados de la investigaci\'on}
% Conocimiento generado en el cumplimiento de cada uno de los
% objetivos. 

\section{Resultados esperados - Productos}
% Evidencian el logro en cuanto a generación de nuevo conocimiento,
% fortalecimiento de capacidades científicas y apropiación social del
% conocimiento, incluir indicadores verificables y medibles acordes
% con los objetivos y alcance del proyecto. 

%(Productos, haciendo particular énfasis en los asociados con
%generación de nuevo conocimiento, con el fortalecimiento de
%capacidades científicas y con la apropiación social del
%conocimiento). Se deben considerar los productos relacionados en el
%documento conceptual que hace parte de la Medición de Grupos de
%Investigación, Desarrollo Tecnológico y/o Innovación, 20132 (ver
%anexo 10). 
 
\subsection{Productos resultado de actividades de Generación Nuevo
  Conocimiento} 

\subsubsection{Art\'iculos de investigaci\'on A1, A2, B y C}
% Artículos en revistas
%indexadas en los índices y bases mencionados en la Tabla I del ANEXO
%1. 

\subsection{Productos resultado de actividades de Desarrollo
  Tecnol\'ogico e Innovaci\'on} 
%Productos tecnológicos certificados o validados
%Diseño industrial, esquema de circuito integrado, software, planta
%piloto y prototipo industrial. Los requerimientos son mencionados en
%la Tabla VII del ANEXO 1. 

\subsection{Productos resultado de actividades de Apropiación Social
  del Conocimiento} 

\subsubsection{Estrategias pedag\'ogicas para el fomento de la CTI}
%Programa/Estrategia pedagógica de fomento a la CTI. Incluye la
%formación de redes de fomento de la apropiación social del
%conocimiento. Los requerimientos son mencionados en la Tabla XII del
%ANEXO 1. 

\subsubsection{Comunicaci\'onn social del conocimiento}
%Estrategias de comunicación del
%conocimiento, generación de contenidos impresos, multimedia y
%virtuales Los requerimientos son mencionados en la Tabla XIII del
%ANEXO 1. 

\subsubsection{Circulaci\'on de conocimiento especializado}

%Eventos científicos y participación en redes de conocimiento,
%documentos de trabajo (working papers), boletines divulgativos de
%resultado de investigación, ediciones de revista científica o de
%libros resultado de investigación e informes finales de
%investigación. Los requerimientos son mencionados en la Tabla XIV del
%ANEXO 1. 

\subsection{Productos de actividades relacionadas con la Formación de Recurso
Humano para la CTI }
%Tesis de Doctorado
%Dirección o co-dirección o asesoría de Tesis de Doctorado, se
%diferencian las t%esis con reconocimiento de las aprobadas. Los
%requerimientos son mencionados en la Tabla XVI del ANEXO 1. 

%Trabajo de grado de Maestría
%Dirección o co-dirección o asesoría de Trabajo de grado de maestría,
%se diferencian los trabajos con reconocimiento de los aprobados. Los
%requerimientos son mencionados en la Tabla XVI del ANEXO 1. 

%Trabajo de grado de Pregrado Dirección o co-dirección o asesoría de
%Trabajo de grao pregrado, se diferencian los trabajos con
%reconocimiento de los aprobados. Los requerimientos son mencionados
%en la Tabla XVI del ANEXO 1. 

\section{Trayectoria del equipo de investigaci\'on}
%Incluir el estado actual de investigación del equipo que conforma la
%propuesta, así como las perspectivas de investigación dentro de la
%temática enmarcada en el proyecto propuesto. 


\section{Posibles evaluadores}
%Identificar nombre y coordenadas de contacto de expertos en la
%temática de investigación a nivel nacional e internacional. 

\noindent
Octavio Valenzuela. Email {\texttt{octavio at astro unam mx}} (UNAM, M\'exico)\\
Nelson Padilla. Email {\texttt{npadilla at astro puc cl}} (PUC, Chile)\\
Patricia Tissera. Email {\texttt{patricia at iafe uba ar}} (UBA, Argentina)\\

\section{Cronograma}
%Distribución de actividades a lo largo del tiempo de ejecución del
%proyecto. Asociar a cada actividad el o los objetivos (numerados)
%relacionados con estos. 

\section{Impacto ambiental}
%Los proyectos de investigación deben incluir una reflexión
%responsable sobre los efectos positivos o negativos que puedan tener
%sobre el medio natural y la salud humana en el corto, mediano y largo
%plazo. 


En la Tabla 1 presentamos un resumen del presupuesto detallado
solicitado a Colciencias. Los items que destacan son los siguientes

\begin{itemize}

\item Apoyo para la pasant\'ia de un estudiante de doctorado en el
  Lawrence Berkeley Laboratory por un per\'iodo de seis meses.
\item Apoyo para la visita de colaboraci\'on de un asistente postdoctoral en el
  Lawrence Berkeley Laboratory por dos per\'iodos de dos meses.
\item Apoyo a viajes cortos (2 semanas) para un profesor, durante los
  tres a\~nos del proyecto. 
\item Compra de una cuchilla de procesadores y almacenamiento de 1TB
  con backup.
\end{itemize}

\begin{tabular}{|l |p{4.5cm}| c |c |c|}\hline
 & Actividad & Costo a\~no 1 & Costo a\~no 2 & Costo a\~no 3\\\hline
1. & Viaje a una escuela internacional de formaci\'on Estudiantes de
Doctorado \#1 y \#2 & & 8.000 & 8.000\\\hline
1. & Viaje a un congreso nacional para los Estudiantes de
Doctorado \#1 y \#2 & & 2.000 & 2.000 \\\hline
2. & Compra de almacenamiento. 2TB con backup & 10.000 & & \\\hline
3. & Pasant\'ia Estudiante de Doctorado \#2. (6 meses)& &  & 36.000\\\hline
4. & Viajes Investigador postdoctoral (1 meses)& & 10.000 & 10.000\\\hline
5. & Viajes internacionales profesor (1 mes) & & 10.000 & 10.000 \\ \hline
6. & Publicaciones &  & 7.000 & 7.000\\\hline 
& {\bf Total Anual} & 10.000 & 37.000 & 73.000\\\hline
& {{\bf Total Proyecto}} & \multicolumn{3}{|c|}{120.000}\\\hline
\end{tabular}


Contrapartidad Universidad de los Andes


\begin{tabular}{|l |p{4.5cm}| c |c |c|}\hline
& Actividad & Costo a\~no 1 & Costo a\~no 2 & Costo a\~no 3\\\hline
1. & Viaje a una escuela internacional de formaci\'on Estudiantes de
Doctorado \#1 y \#2 & 8.000 & & \\\hline
1. & Viaje a un congreso nacional para los Estudiantes de
Doctorado \#1 y \#2 & 2.000 & & \\\hline
5. & Viajes internacionales profesor (4 semanas) & 10.000 & & \\ \hline
6. & Publicaciones &  7.000 & & \\\hline 
2. & Salario B\'asico Jaime E. Forero-Romero (2014) & 20.000 & 20.000
& 20.000 \\\hline  

3. & Pasant\'ia Estudiante de Doctorado \#1 (1 mes) & 36.000 &
36.000 & \\\hline

4. & Salario B\'asico T\'ecnico DSIT (2014) & 5.000 & 5.000
& 5.000 \\\hline  

5. & Visita y Curso de Stefan Gottloeber (1 meses) & 8.000 & & \\\hline

& {\bf Total Anual} & 96.000 & 61.000 & 25.000\\\hline
& {{\bf Total Proyecto}} & \multicolumn{3}{|c|}{185.000}\\\hline

\end{tabular} 

La dedicaci\'on al proyecto es de 15\% del tiempo para
profesores. El costo mensual del salario se compone de un valor
b\'asico (X.XXX) m\'as un factor prestacional (X.XXX) Se asumen
incrementos salariales anuales del 4\%. 

La dedicaci\'on al proyecto es de 15\% para los t\'ecnicos encargados
de mantener e instalar los recursos de computaci\'on de alto
rendimiento. 


\section{Bibliograf\'ia}
%Fuentes bibliográficas empleadas en cada uno de los ítems del
%proyecto. Se hará referencia únicamente a aquellas fuentes empleadas
%en el suministro de la información del respectivo proyecto. No se
%incluirán referencias que no se citen. Las citas, en cada uno de los
%campos del formulario, se harán empleando el número de la referencia
%dentro de paréntesis cuadrados (p. ej. [1]). 

\bibliography{referencias}{}
\bibliographystyle{plain}


\end{document}
