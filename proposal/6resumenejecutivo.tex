
\section{Resumen ejecutivo}
% Información mínima necesaria para comunicar de manera precisa los
% contenidos y alcances del proyecto. 

La {\bf cosmología observacional} entró en una época dorada con la medición
de las anisotropías de la radiación cósmica de fondo (Premio Nobel de
Física 2006) y la medición de la expansión acelerada del Universo
(Premio Nobel de Física 2011).  
Hoy en d\'ia una de las fronteras de la investigación en esta \'area 
es la obtenci\'on de mejores mediciones de la historia de expansión
acelerada del Universo. 

El objetivo de estas mediciones es restringir los  modelos de la
llamada Energ\'ia Oscura, posible responsable de este efecto. 
Una de las t\'ecnicas observacionales que se utiliza con ese
prop\'osito es la detecci\'on del pico de Oscilaciones Ac\'usticas de
Bariones (OAB) que requiere la medici\'on de las \emph{posiciones} de
millones de galaxias \cite{Eisenstein2005}. 
Otros m\'etodos usan informaci\'on sobre las \emph{velocidades}
peculiares (i.e. velocidades que no es\'an asociadas al flujo de
Hubble) de las galaxias y las distorsiones que estas generan en las
observaciones para cuantificar el crecimiento de estructura a gran
escala \cite{Scoccimarro2004}. 

Las {\bf simulaciones computacionales} juegan un rol central en estos
esfuerzos observacionales para medir el Universo.
Las simulaciones son necesarias para traducir las premisas te\'oricas
en cantidades observables. 
Es decir, son un puente entre la teor\'ia y
la  observaci\'on. 
Hoy en d\'ia,  las simulaciones tambi\'en sirven para preparar una
nueva campa\~na observacional.  
El objetivo es simular todo antes de empezar a medir. 


{\bf La presente propuesta tiene como objetivo hacer investigaci\'on en
Cosmolog\'ia Computacional como fundamento de actividades en
Cosmolog\'ia Observacional.}

En el frente de cosmolog\'ia computacional proponemos realizar una serie de
simulaciones de la distribuci\'on de materia en el Universo a gran
escala sin precedentes en el pa\'is. 
Para la preparaci\'on de estas simulaciones haremos uso de las
m\'aquinas de las Instalaciones para C\'omputo de Alto Rendimiento de
la Universidad de los Andes, las cuales ser\'an instaladas y puestas
en funcionamiento durante el segundo semestre del 2014.

Vamos a usar estas simulaciones de manera principal para fundamentar
nuestro trabajo en Cosmolog\'ia Observacional en dos colaboraciones
internacionales. 

La primera colaboraci\'on, con el Lawrence Berkeley National Laboratory,
se trata de contribuir en la preparaci\'on del experimento DESI (Dark
Energy Spectroscopic Instrument). 
DESI es una colaboraci\'on internacional de m\'as de 100 cient\'ificos
que construir\'a el experimento de siguiente generaci\'on para medir
la historia de expansi\'on del Universo haciendo un mapa de la
distribuci\'on de 25 millones de galaxias, 10 veces m\'as de lo que ha
sido observado hasta la fecha \cite{DESI}. 
En particular, esperamos durante la duraci\'on del presente proyecto
con COLCIENCIAS aportar al dise\~no y optimizaci\'on de la estrategia
de las observaciones.  

La segunda colaboraci\'on, con el Korean Insitute for Advanced Studies
(KIAS), trata de la la restricci\'on de par\'ametros cosmol\'ogicos a
partir de la anisotrop\'ia de la distribuci\'on de galaxias. 
En una primera etapa esperamos usar las simulaciones para optimizar los
m\'etodos estad\'isticos de medici\'on del efecto buscado. 
En una segunda etapa vamos a aplicar este conocimiento sobre los datos
observacionales del proyecto SDSS-III (Sloan Digital Sky Survey),
algo posible gracias a la participaci\'on de KIAS en ese proyecto.

Los proyectos posibles con la serie de simulaciones que vamos a
realizar abren la puerta a otro tipo de estudios sobre la estructura
del Universo a gran escala \cite{Tweb,Vweb} y la cuantificaci\'on de
la abundancia de colisiones extremas en el Universo
\cite{Bullets2010,Bullets2014}. 
Adicionalmente, estos datos se har\'an disponibles a toda la comunidad
(colombiana e internacional) para realizar diferentes tipos de
estudios y an\'alisis \cite{Multidark}.  
En el curso de este proyecto tambi\'en vamos a organizar una escuela
internacional de un mes, con expertos internacional, para formar
investigadores en Colombia y en la regi\'on andina que est\'enn en capacidad de
utilizar recursos computacionales avanzados para resolver problemas en
cosmolog\'ia computacional y observacional.
