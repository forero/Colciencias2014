\documentclass{report}
\begin{document}


\section*{Equipos}
%Aquellos necesarios para el desarrollo del proyecto, los cuales
%pueden ser adquiridos a cualquier título. La financiación para compra
%de equipos nuevos deberá estar sustentada en la estricta necesidad de
%los mismos para el desarrollo del proyecto.  

\section*{Materiales e Insumos}
% Adquisición de insumos, bienes fungibles y demás elementos
% necesarios para el desarrollo de algunas actividades
% previstas. Deben presentarse a manera de listado detallado agrupado
% por categorías sobre las cuales debe hacerse una justificación de su
% necesidad y cantidad (Ej. consumibles, reactivos, herramientas,
% elementos de protección, controles e instrumentación accesoria,
% material biológico, audiovisual, de laboratorio y de campo, etc). 

\section*{Servicios t\'ecnicos}
%Contrataciones que se hacen para la prestación de servicios
%especializados y cuya necesidad esté suficientemente justificada, por
%ejemplo: ensayos, pruebas, análisis de laboratorio y
%caracterizaciones, etc. Estos no deben incluirse en los gastos de
%personal. 

\section*{Software}
% Adquisición de licencias de software especializado para las
% actividades de CTI propias del desarrollo del proyecto. Su necesidad
% y cantidad debe soportarse en justificaciones técnicas
% detalladas. No se considerará financiable dentro de este rubro
% software de uso cotidiano, como por ejemplo procesadores de texto,
% hojas electrónicas o sistemas operativos. 

\section*{Salidas de campo}
%Costos asociados al levantamiento de información en campo, desde
%fuentes primarias o secundarias, para la consecución de los objetivos
%del proyecto. 

\section*{Viajes}
% Se refiere a los gastos de transporte (pasajes nacionales e
% internacionales) y viáticos relacionados con las actividades
% propuestas en el componente científico-técnico del proyecto
% (capacitaciones, estancias cortas en instituciones académicas
% nacionales o extranjeras, presentación de ponencias en eventos
% especializados, etc.) y que son estrictamente necesarios para la
% ejecución exitosa del proyecto y la generación de productos y
% resultados. 

\section*{Eventos Acad\'emicos}
%Gastos ocasionados por la organización y divulgación de eventos
%(paneles, simposios, talleres, seminarios, congresos, ferias de CTI,
%etc.) que permitan retroalimentar o presentar productos y resultados
%del proyecto. 

\section*{Publicaciones}
% Costos de publicación de artículos científicos en revistas indexadas
% con un alto factor de impacto. Costos asociados a la publicación de
% libros, manuales, videos, cartillas, posters, etc. que presenten los
% resultados del Programa y sirvan como estrategia de divulgación o
% apropiación social de los resultados de la investigación. 


\section*{Seguimiento y Evoluaci\'on}
%Corresponde al 3% de la sumatoria de los rubros con cargo a
%COLCIENCIAS, incluido el rubro de Gastos de operación. Los recursos
%de este rubro se destinarán a las actividades de seguimiento y
%evaluación de la ejecución del proyecto por parte de COLCIENCIAS. 

\section{Gastos de operaci\'on} 
% Hasta el 10% de las sumatoria de los rubros con cargo a COLCIENCIAS,
% sin incluir el rubro de seguimiento y evaluación. 


\section*{Personal cient\'ifico para actividades de CTI}
%♦ Personal con formación científica y técnica, que cuenta con título
%profesional y/o de posgrado (maestría, doctorado), y vinculación de
%postdoctorados que estarán a cargo de las actividades investigativas
%propias de la ejecución del proyecto según el planteamiento
%científico-técnico. 

%♦ Personal con formación en carreras técnicas y tecnológicas con
% capacidades para apoyar la ejecución de actividades de CTI.} 


\end{document}
